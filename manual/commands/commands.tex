\chapter{Simulation input options}

\section{Input sections}

\newpage
\begin{lstlisting}[name=listing,escapechar=!, caption={Structure and input sections of the \texttt{simulation.json} file.},captionpos=b]
{
  "SimulationType" : "MolecularDynamics", !\tikzmark{bgnShifted}!
  "NumberOfCycles" : 100000,
  "NumberOfInitializationCycles" : 1000,
  "NumberOfEquilibrationCycles" : 10000,
  "PrintEvery" : 1000, !\tikzmark{trmShifted}!

  "Systems" : [                        !\tikzmark{bgnBrace}!
    {
      "Type" : "Framework",
      "Name" : "Cu-BTC",
      "NumberOfUnitCells" : [1, 1, 1],
      "ChargeMethod" : "Ewald",
      "ExternalTemperature" : 323.0,
      "ExternalPressure" : 1.0e4,
      "OutputPDBMovie" : false,
      "SampleMovieEvery" : 10
    }
  ],                                   !\tikzmark{trmBrace}!

  "Components" : [                     !\tikzmark{bgnOther}!
    {
      "Name" : "CO2",
      "FugacityCoefficient" : 1.0,
      "TranslationProbability" : 0.5,
      "RotationProbability" : 0.5,
      "ReinsertionProbability" : 0.5,
      "SwapProbability" : 0.0,
      "WidomProbability" : 0.0,
      "CreateNumberOfMolecules" : 20
    }
  ]                                    !\tikzmark{trmOther}!
}
\end{lstlisting}

\begin{tikzpicture}[overlay, remember picture]
    \drawBrace[.6em]{bgnShifted}{trmShifted}{General options.};
    \drawBrace{bgnBrace}{trmBrace}{System options};
    \drawBrace{bgnOther}{trmOther}{Component options};
\end{tikzpicture}

\section{General options}

\subsection{Simulation types}

\begin{itemize}
\item{\verb+"SimulationType" : "MonteCarlo"+}\\
Starts the Monte Carlo part of RASPA. The particular ensemble is not specified but implicitly deduced from
the specified Monte Carlo moves. Note that a MD-move can be used for hybrid MC/MD.
\item{\verb+"SimulationType" : "MolecularDynamics"+}\\
Starts the Molecular Dynamics part of RASPA. The ensemble must be explicitly specified.
\end{itemize}

\subsection{Simulation duration}

\begin{itemize}
\item{\verb+"NumberOfCycles" : integer+}\\
The number of cycles for the production run.
For Monte Carlo a cycle consists of $N$ steps, where $N$ is the amount of
molecules with a minimum of 20 steps. This means that on average during each cycle on each molecule a
Monte Carlo move has been attempted (either successful or unsuccessful). For MD the number of cycles
is simply the amount of integration steps.
\item{\verb+"NumberOfInitializationCycles": integer+}\\
The number of cycles used to initialize the system using Monte Carlo. This can be used for both Monte Carlo
as well as Molecular Dynamics to quickly equilibrate the positions of the atoms in the system.
\item{\verb+"NumberOfEquilibrationCycles" : integer+}\\
For Molecular Dynamics  it is the number of MD steps to equilibrate the velocities in the systems. After this
equilibration the production run is started. For Monte Carlo, in particular CFCMC, the equilibration-phase is used
to measure the biasing factors.
\end{itemize}

\subsection{Restart and crash-recovery}

\begin{itemize}
\item{\verb+"RestartFile" : boolean+}\\
Reads the positions, velocities, and force from the directory `RestartInitial'. Any creation of molecules
in the `simulation.input' file will be in addition and after this first read from file. This is useful to
load initial positions of cations for example, and after that create adsorbates. The restart file
is written at `PrintEvery' intervals.
\item{\verb+"ContinueAfterCrash" : boolean+}\\
Write a binary file containing the complete status of the program.
The file name is `binary\_restart.dat' and is located in the directory `CrashRestart'.
With this option to \verb+true+ the presence of this file will result in continuation from
the point where the program was at the moment of outputting this file.
The file can be quite big (several hundreds of megabytes) and will be outputted every
`WriteBinaryRestartFileEvery' cycles.
\item{\verb+"WriteBinaryRestartFileEvery" : integer+}\\
The output frequency (i.e. every [integer] cycles) of writing the crash-recovery file.
\end{itemize}

\subsection{Printing options}
\begin{itemize}
\item{\verb+"PrintEvery" : integer+}\\
Prints the loadings (when a framework is present) and energies every [integer] cycles. For MD information
like energy conservation and stress are printed.
\end{itemize}


\section{System options}

\subsection{Operating conditions and thermostat/barostat-parameters}

\begin{itemize}
\item{\verb+"ExternalTemperature" : floating-point-number+}\\
The external temperature in Kelvin for the system. Default: 298 K.
\item{\verb+"ExternalPressure" : floating-point-number+}\\
The external pressure in Pascal for the system. Default: 0 Pa.
\item{\verb+"ThermostatChainLength" : integer+}\\
The length of the chain to thermostat the system. Default: 5.
\item{\verb+"NumberOfYoshidaSuzukiSteps" : integer+}\\
The number of Yoshida/Suzuki multiple timesteps. Default: 5.
\item{\verb+"TimeScaleParameterThermostat" : floating-point-number+}\\
The time scale on which the system thermostat evolves. Default: 0.15 ps.
\end{itemize}

\subsection{Box/Framework options}

\begin{itemize}
\item{\verb+"Type" : string+}\\
Sets the system type. The type string can be:
  \begin{itemize}
  \item{\verb+"Box"+}\\
    Sets the system to a simulation cell where the lengths and angles of the cell can be explicitely be specified.
  \item{\verb+"Framework"+}\\
    Set the system to type `Framework'. The cell lengths and cell angles follows from the specified framework file.
  \end{itemize}
\item{\verb+"BoxLengths" : [floating-point-number, floating-point-number, floating-point-number]+}\\
The cell dimensions of rectangular box of system in Angstroms.
\item{\verb+"BoxAngles" : [floating-point-number, floating-point-number, floating-point-number]+}\\
The cell angles of rectangular box of system in Degrees.
\item{\verb+"Name" : string+}\\
Loads the framework with name \verb+string.cif+.
\item{\verb+"NumberOfUnitCells" : [integer, integer, integer]+}\\
The number of unit cells in \verb+x+, \verb+y+, and \verb+z+ direction for the system. 
The super-cell will contain the unit cells, and periodic boundary conditions
will be applied on the super-cell level (\emph{not} on a unit cell level).
\item{\verb+"HeliumVoidFraction" : floating-point-number+}\\
Sets the void fraction as measured by probing the structure with helium a room temperature. 
This quantity has to be obtained from a separate simulation 
and is essential to compute the \emph{excess}-adsorption during the simulation.
\end{itemize}

\subsection{Force field definitions}

\begin{itemize}
\item{\verb+"ForceField" : string+}\\
Reads in the force field file \verb+string.json+,
Note that if this file is in the working directory then this will be read and used instead of:
\begin{verbatim}
${RASPA_DIR}/simulations/share/raspa3/forcefield/string/force_field.json
\end{verbatim}
\item{\verb+"CutOffVDW" : floating-point-number+}\\
The cutoff of the Van der Waals potentials. Interactions longer then this distance are omitted from the
energy and force computations. The potential can either be shifted to zero at the cutoff, or interactions
can just neglected after the cut off, or the remainder of the potential energy can be approximated using
tail corrections. This is specified in the force field files and can be specified globally or
for each interaction individually.
\item{\verb+"CutOffCoulombic" : floating-point-number+}\\
The cutoff of the charge-charge potential. The potential is truncated at the cutoff.
No tail-corrections are (or can be) applied. The only way to include the long-range part is to use `ChargeMethod Ewald'.
The parameter is also used in combination with the
Ewald precision to compute the number of wave vectors and Ewald parameter $\alpha$.
For the Ewald summation using rather large unit cells, a charge-charge cutoff of about half the smallest box-length would be advisable
in order to avoid the use of an excessive amount of wave-vectors in Fourier space. For non-Ewald methods the cutoff should be as large
as possible (greater than about 30 \AA).
\item{\verb+"ChargeMethod" : string+}
Sets the method to compute charges. The string can be:
  \begin{itemize}
  \item{\verb+"None"+}\\
    Skips the entire charge calculation and should only be used when all adsorbates do not contain any charges.
  \item{\verb+"Ewald"+}\\
    Switches on the Ewald summation for the charge calculation.
  \end{itemize}
\item{\verb+"UseChargesFromCIFFile" : boolean+}
\end{itemize}

\subsection{System MC-moves}

\begin{itemize}
\item{\verb+"VolumeChangeProbability" : floating-point-number+}\\
The probability per cycle to attempt a volume-change. 
Rigid molecules are scaled by center-of-mass, while flexible molecules and the framework is atomically scaled.
\item{\verb+"GibbVolumeChangeProbability" : floating-point-number+}\\
The probability per cycle to attempt a Gibbs volume-change MC move during a Gibbs ensemble simulation. The total volume of the two boxes
(usually one for the gas phase, one for the liquid phase) remains constant, but the individual volume of the boxes are changed.
The volumes are changed by a random change in $\ln(V_I/V_{II})$.
\end{itemize}

\subsection{Molecular dynamics parameters}

\begin{itemize}
\item{\verb+"TimeStep" : floating-point-number+}\\
The time step in picoseconds for MD integration. Default value: 0.0005 ps (0.5 fs).
\item{\verb+"Ensemble" : string+}\\
Sets the ensemble. The ensemble string can be:
  \begin{itemize}
  \item{\verb+"NVE"+}\\
  The micro canonical ensemble, the number of particle $N$, the volume $V$, and the energy $E$ are constant.
  \item{\verb+"NVT"+}\\
  The canonical ensemble, the number of particle $N$, the volume $V$, and the average temperature $\left\langle T\right\rangle$
  are constant. Instantaneous values for the temperature are fluctuating.
  \end{itemize}
\end{itemize}

\subsection{Options to measure properties}

\subsubsection{Output pdb-movies}
\begin{framed}
\verb+"OutputPDBMovie" : boolean+
\end{framed}
Sets whether or not to output simulation snapshots to pdb movies.\\
Output is written to the directory \verb+movies+.
\begin{itemize}
\item{\verb+"SampleMovieEvery" : integer+}\\
Sample the movie every [integer] cycles. Default: 1.
\end{itemize}

\subsubsection{Histogram of the energy}
\begin{framed}
\verb+"ComputeEnergyHistogram" : boolean+
\end{framed}
Sets whether or not to compute a histogram of the energy for the current system.
For example, during adsorption it keeps track of the total energy, the VDW energy,
the Coulombic energy, and the polarization energy.\\
Output is written to  the directory \verb+energy_histogram+.
\begin{itemize}
\item{\verb+"SampleEnergyHistogramEvery" : integer+}\\
Sample the energy histogram of the system every [integer] cycles. Default: 1.
\item{\verb+"WriteEnergyHistogramEvery" : integer+}\\
Writes the energy histogram of the system every [integer] cycles. Default: 5000.
\item{\verb+"NumberOfBinsEnergyHistogram" : integer+}\\
Sets the number of elements of the histogram. Default: 128.
\item{\verb+"LowerLimitEnergyHistogram" : floating-point-number+}\\
The lower limit of the histogram. Default: -5000.
\item{\verb+"UpperLimitEnergyHistogram" : floating-point-number+}\\
The upper limit of the histogram. Default: 1000.
\end{itemize}

\subsubsection{Histogram of the number of molecules}
\begin{framed}
\verb+"ComputeNumberOfMoleculesHistogram" : boolean+
\end{framed}
Sets whether or not to compute the histograms of the number of molecules for the current system.
In open ensembles the number of molecules fluctuates.\\
Output is written to the directory \verb+number_of_molecules_histogram+.
\begin{itemize}
\item{\verb+"SampleNumberOfMoleculesHistogramEvery" : integer+}\\
Sample the histogram every [integer] cycles. Default: 1.
\item{\verb+"WriteNumberOfMoleculesHistogramEvery" : integer+}\\
Output the histogram every [integer] cycles. Default: 5000.
\item{\verb+"LowerLimitNumberOfMoleculesHistogram" : floating-point-number+}\\
The lower limit of the histograms. Default: 0.
\item{\verb+"UpperLimitNumberOfMoleculesHistogram" : floating-point-number+}\\
The upper limit of the histograms. Default: 200.
\end{itemize}

\subsubsection{Radial Distribution Function (RDF) force-based}
\begin{framed}
\verb+"ComputeRDF" : boolean+
\end{framed}
Sets whether or not to compute the radial distribution function (RDF).\\
Output is written to the directory \verb+rdf+.
\begin{itemize}
\item{\verb+"SampleRDFEvery" : integer+}\\
Sample the rdf every [integer] cycles. Default: 10.
\item{\verb+"WriteRDFEvery" : integer+}\\
Output the rdf every [integer] cycles. Default: 5000.
\item{\verb+"NumberOfBinsRDF" : integer+}\\
Sets the number of elements of the rdf. Default: 128.
\item{\verb+"UpperLimitRDF" : floating-point-number+}\\
The upper limit of the rdf. Default: 15.0.
\end{itemize}


\subsubsection{Radial Distribution Function (RDF) conventional}
\begin{framed}
\verb+"ComputeConventionalRDF" : boolean+
\end{framed}
Sets whether or not to compute the radial distribution function (RDF).\\
Output is written to the directory \verb+conventional_rdf+.
\begin{itemize}
\item{\verb+"SampleConventionalRDFEvery" : integer+}\\
Sample the rdf every [integer] cycles. Default: 10.
\item{\verb+"WriteConventionalRDFEvery" : integer+}\\
Output the rdf every [integer] cycles. Default: 5000.
\item{\verb+"NumberOfBinsConventionalRDF" : integer+}\\
Sets the number of elements of the rdf. Default: 128.
\item{\verb+"UpperLimitConventionalRDF" : floating-point-number+}\\
The upper limit of the rdf. Default: 15.0.
\end{itemize}

\subsubsection{Mean-Squared Displacement (MSD) order-N}
\begin{framed}
\verb+"ComputeMSD" : boolean+
\end{framed}
Sets whether or not to compute the mean-squared displacement (MSD).\\
Output is written to the directory \verb+msd+.
\begin{itemize}
\item{\verb+"SampleMSDEvery" : integer+}\\
Sample the msd every [integer] cycles. Default: 10.
\item{\verb+"WriteMSDEvery" : integer+}\\
Output the msd every [integer] cycles. Default: 5000.
\item{\verb+"NumberOfBlockElementsMSD" : integer+}\\
The number of elements per block of the msd. Default: 25.0.
\end{itemize}

\subsubsection{Density grids}
\begin{framed}
\verb+"ComputeDensityGrid" : boolean+
\end{framed}
Sets whether or not to compute the density grids.\\
Output is written to the directory \verb+density_grids+.
\begin{itemize}
\item{\verb+"SampleDensityGridEvery" : integer+}\\
Sample the density grids every [integer] cycles. Default: 10.
\item{\verb+"WriteDensityGridEvery" : integer+}\\
Output the density grids every [integer] cycles. Default: 5000.
\item{\verb+"DensityGridSize" : [integer, integer, integer]+}\\
Sets the size of the density grids. Default: [128, 128, 128].
\item{\verb+"DensityGridPseudoAtomsList" : [string]+}\\
The list of pseudo-atoms for which to compute the density grids.
\end{itemize}



\section{Component options}

\subsection{Component properties}

\begin{itemize}
\item{\verb+"Name" : string+}\\
The descriptive name of the component.
\item{\verb+"MolFraction" : floating-point-number+}\\
The mol fraction of this component in the mixture. The values can be specified relative to other components, as the fractions are normalized afterwards.
The partial pressures for each component are computed from the total pressure and the mol fraction per component.
\item{\verb+"FugacityCoefficient" : floating-point-number+}\\
The fugacity coefficient for the current component. For values 0 (or by not specifying this line), the fugacity coefficients are automatically computed using the Peng-Robinson
equation of state. Note the critical pressure, critical temperature, and acentric factor need to be specified in the molecule file.
\item{\verb+"IdealGasRosenbluthWeight" : floating-point-number+}\\
The ideal Rosenbluth weight is the growth factor of the CBMC algorithm for a single chain in an empty box. The value only depends on temperature and therefore needs to be computed
only once. For adsorption, specifying the value in advance is convenient because the applied pressure does not need to be corrected afterwards (the Rosenbluth weight corresponds to a shift
in the chemical potential reference value, and the chemical potential is directly obtained from the fugacity). For equimolar mixtures this is essential.
\item{\verb+"CreateNumberOfMolecules" : integer+}\\
The number of molecule to create for the current component. 
Note these molecules are \emph{in addition} to anything read in by using a restart-file. Usually, when the restart-file
is used the amount here should be put back to zero. A warning, putting this value unreasonably high results in an infinite loop. 
The routine accepts molecules that are grown causing no overlap (energy smaller than `EnergyOverlapCriteria'). 
Also the initial starting configurations are far from optimal and substantial equilibration is needed to reduce the energy.
However, the CBMC growth is able to reach very high densities.
\item{\verb+"BlockingPockets" : [[3 x floating-point-number, floating-point-number]]+}\\
Block certain pockets in the simulation volume. The growth of a molecule is not allowed in a blocked pocket. 
A typical example is the sodalite cages in FAU and LTA-type zeolites,
these are not accessible to molecules like methane and bigger.
The pockets are specified as a list of 4 floating points numbers: the $s_x$, $s_y$, $s_z$ fractional positions and 
a radius in Angstrom.

For example, blocking pockets for ITQ-29 for small molecules are specified as:
\begin{verbatim}
"BlockingPockets" : [
           [0.0,       0.0,        0.0,       4.0],
           [0.5,       0.0,        0.0,       0.5],
           [0.0,       0.5,        0.0,       0.5],
           [0.0,       0.0,        0.5,       0.5]
         ]
\end{verbatim}
\end{itemize}

\subsection{Component MC-moves}

\begin{itemize}

\item{\verb+"TranslationProbability" : floating-point-number+}\\
The relative probability to attempt a translation move for the current component. 
A random displacement is chosen in the allowed directions (see `TranslationDirection').
Note that the internal configuration of the molecule is unchanged by this move. 
The maximum displacement is scaled during the simulation to achieve an acceptance
ratio of 50\%.
\item{\verb+"RandomTranslationProbability" : floating-point-number+}\\
The relative probability to attempt a random translation move for the current component. 
The displacement is chosen such that any position in the box can reached. It is therefore
similar as reinsertion, but `reinsertion' changes the internal conformation of a molecule and uses biasing.
\item{\verb+"RotationProbability" : floating-point-number+}\\
The relative probability to attempt a random rotation move for the current component. 
The rotation is around the starting bead. A random vector on a sphere
is generated, and the rotation is random around this vector.
\item{\verb+"ReinsertionProbability" : floating-point-number+}\\
The relative probability to attempt a full reinsertion move for the current component. 
Multiple first beads are chosen, and one of these is selected according to its Boltzmann weight.
The remaining part of the molecule is grown using biasing. 
This move is very useful, and often necessary, to change the internal configuration of flexible molecules.
\item{\verb+"SwapProbability" : floating-point-number+}\\
The relative probability to attempt a insertion or deletion move. 
Whether to insert or delete is decided randomly with a probability of 50\% for each.
The swap move imposes a chemical equilibrium between the system and 
an imaginary particle reservoir for the current component. 
The move starts with multiple first bead, and
grows the remainder of the molecule using biasing.
\item{\verb+GibbsSwapProbability" : floating-point-number+}\\
The relative probability to attempt a Gibbs swap MC move for the current component. 
The `GibbsSwapMove' transfers a randomly selected particle from one box to the other
(50\% probability to transfer a particle from box I to II, an 50\% visa versa).
\item{\verb+"WidomProbability" : floating-point-number+}\\
The relative probability to attempt a Widom particle insertion move for the current component. 
The Widom particle insertion moves measure the chemical potential
and can be directly related to Henry coefficients and heats of adsorption.
\end{itemize}
